\section{Conclusion}
L'objectif visé une architecture de serveur de jeu massivement multijoueurs dans le cloud est de réunir tous les joueurs d'un même jeu dans un même monde simulé sans bordures pour permettre une immersion plus grande dans le jeu. Le but de ce stage est donc de fournir une plateforme de simulation pour de ces architectures. Il faudra aussi tester des solutions permettant une gestion efficace de ces serveurs et de les comparer à l'aide des métriques exposées dans la partie IV. Cependant, l'objectif du stage n'est pas de fournir une solution complète de gestion de serveur de jeu dans le cloud, comme RTF~\cite{rtf_middleware_development} le fait pour les environnements multi-serveurs. Par ailleurs, la gestion des fautes (réplication) et de la triche ne seront pas envisagées durant le stage.\\

Pour le simulateur, nous allons utiliser Simgrid~\cite{simgrid} car il possède de nombreux avantages. Tout d'abord, il gère l'utilisation de VMs dans le cloud et simule fidèlement le réseau. Simgrid possède des bindings en Java qui permettent de programmer plus rapidement des solutions et d'utiliser un plus haut niveau d'abstraction grâce aux outils préexistants en Java.\\

Pour permettre de mieux visualiser l'effet des algorithmes mis en \oe{}uvre dans l'architecture cloud, nous allons fournir divers outils graphiques permettant de visualiser les effets des algorithmes, notamment avec une visualisation de la carte du monde montrant le déplacement des joueurs ainsi que l'évolution du partitionnement et des courbes montrant l'évolution des métriques en fonction des données du système (temps de la simulation, nombre de joueur connecté...). Ceci permettra de proposer des algorithmes plus efficaces de façon plus intuitive en visualisant facilement les goulots d'étranglements au niveau de la répartition des serveurs.
